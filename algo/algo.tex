%%%%%%%%%%%%%%%%%%%%%%%%%%%%% Define Article %%%%%%%%%%%%%%%%%%%%%%%%%%%%%%%%%%
\documentclass{article}
%%%%%%%%%%%%%%%%%%%%%%%%%%%%%%%%%%%%%%%%%%%%%%%%%%%%%%%%%%%%%%%%%%%%%%%%%%%%%%%

%%%%%%%%%%%%%%%%%%%%%%%%%%%%% Using Packages %%%%%%%%%%%%%%%%%%%%%%%%%%%%%%%%%%
\usepackage{geometry}
\usepackage{graphicx}
\usepackage{amssymb}
\usepackage{amsmath}
\usepackage{amsthm}
\usepackage{empheq}
\usepackage{mdframed}
\usepackage{booktabs}
\usepackage{lipsum}
\usepackage{graphicx}
\graphicspath{ {./images/} }
\usepackage{color}
\usepackage{psfrag}
\usepackage{comment}
\usepackage{pgfplots}
\usepackage{sidecap}
\usepackage{bm}
\usepackage{physics}
\usepackage{wrapfig}
\usepackage{listings}
\def\magyarOptions{suggestions=no} % nem tudom miért de kell ez a sor
\usepackage[magyar, english]{babel}
%%%%%%%%%%%%%%%%%%%%%%%%%%%%%%%%%%%%%%%%%%%%%%%%%%%%%%%%%%%%%%%%%%%%%%%%%%%%%%%

%%%%%%%%%%%%%%%%%%%%%%%%%% Page Setting %%%%%%%%%%%%%%%%%%%%%%%%%%%%%%%%%%%%%%%
\geometry{a4paper}


%%%%%%%%%%%%%%%%%%%%%%%%%% Define some useful colors %%%%%%%%%%%%%%%%%%%%%%%%%%
\definecolor{ocre}{RGB}{243,102,25}
\definecolor{mygray}{RGB}{243,243,244}
\definecolor{deepGreen}{RGB}{26,111,0}
\definecolor{shallowGreen}{RGB}{235,255,255}
\definecolor{deepBlue}{RGB}{61,124,222}
\definecolor{shallowBlue}{RGB}{235,249,255}
%%%%%%%%%%%%%%%%%%%%%%%%%%%%%%%%%%%%%%%%%%%%%%%%%%%%%%%%%%%%%%%%%%%%%%%%%%%%%%%

%%%%%%%%%%%%%%%%%%%%%%%%%% Define an orangebox command %%%%%%%%%%%%%%%%%%%%%%%%
\newcommand{\orangebox}[1]{\fcolorbox{ocre}{mygray}{\hspace{1em}#1\hspace{1em}}}
%%%%%%%%%%%%%%%%%%%%%%%%%%%%%%%%%%%%%%%%%%%%%%%%%%%%%%%%%%%%%%%%%%%%%%%%%%%%%%%

%%%%%%%%%%%%%%%%%%%%%%%%%%%% English Environments %%%%%%%%%%%%%%%%%%%%%%%%%%%%%
\newtheoremstyle{mytheoremstyle}{3pt}{3pt}{\normalfont}{0cm}{\rmfamily\bfseries}{}{1em}{{\color{black}\thmname{#1}~\thmnumber{#2}}\thmnote{\,--\,#3}}
\newtheoremstyle{myproblemstyle}{3pt}{3pt}{\normalfont}{0cm}{\rmfamily\bfseries}{}{1em}{{\color{black}\thmname{#1}~\thmnumber{#2}}\thmnote{\,--\,#3}}
\theoremstyle{mytheoremstyle}
\newmdtheoremenv[linewidth=1pt,backgroundcolor=shallowGreen,linecolor=deepGreen,leftmargin=0pt,innerleftmargin=20pt,innerrightmargin=20pt,]{theorem}{Theorem}[section]
\theoremstyle{mytheoremstyle}
\newmdtheoremenv[linewidth=1pt,backgroundcolor=shallowBlue,linecolor=deepBlue,leftmargin=0pt,innerleftmargin=20pt,innerrightmargin=20pt,]{definition}{Definition}[section]
\theoremstyle{myproblemstyle}
\newmdtheoremenv[linecolor=black,leftmargin=0pt,innerleftmargin=10pt,innerrightmargin=10pt,]{problem}{Problem}[section]
%%%%%%%%%%%%%%%%%%%%%%%%%%%%%%%%%%%%%%%%%%%%%%%%%%%%%%%%%%%%%%%%%%%%%%%%%%%%%%%

%%%%%%%%%%%%%%%%%%%%%%%%%%%%%%% Plotting Settings %%%%%%%%%%%%%%%%%%%%%%%%%%%%%
\usepgfplotslibrary{colorbrewer}
\pgfplotsset{width=8cm, compat=1.9}
%%%%%%%%%%%%%%%%%%%%%%%%%%%%%%%%%%%%%%%%%%%%%%%%%%%%%%%%%%%%%%%%%%%%%%%%%%%%%%%

%%%%%%%%%%%%%%%%%%%%%%%%%%%%%%% Title & Author %%%%%%%%%%%%%%%%%%%%%%%%%%%%%%%%
\title{Cím}
\author{Princzes
Barnabás}
%%%%%%%%%%%%%%%%%%%%%%%%%%%%%%%%%%%%%%%%%%%%%%%%%%%%%%%%%%%%%%%%%%%%%%%%%%%%%%%


\begin{document}
\selectlanguage{magyar}
\maketitle
\begin{abstract}
    Lent hagytam a füzetem és jövő héten doga szóval time to kick ass!
    Írjunk egy jó kis jegyzetet.
\end{abstract}
\section{Alapok}
Van a \{$B$\}$A$\{$K$\} azaz a Bemeneti
feltétel (egy logikai állítás) Algoritmus és Kimeneti feltétel
(szintén egy logikai állítás).
\newline
Legyen $A$ algoritmus. $e_1,\ldots,e_m$ elemi műveletek az algoritmusban
és $t_i$ az edott $e_i$-hez tartozó időigény. Az algoritmus tényleges futási ideje
$T(A,x)$ ahol x egy bemenet és a bemenet mérete $|x|$ például
(tömb, halmaz\ldots) esetében az elemek száma.
Az $e_i$ és $t_i$ és $|x|$ együtt a bonyolultság mértéke.
\section{Esetek}
\begin{itemize}
    \item Legjobb eset: \[T_{lj} = \min \{T(A,x):|x|=n\}\]
    \item Legrosszabb eset: \[T_{lr} = \max\{T(A,x):|x|=n\}\]
    \item Átlagos eset: Legyen $\Pr(x)$ annak
          a valószínűsége, hogy épp $x$ lesz
          $A$ algoritmus bemenete, ekkor
          \newline \[T_a(A,n) = \sum_{|x|=n}\Pr(x)T(A,x)\]
\end{itemize}
\begin{comment}
\begin{lstlisting}[language=Java]
public static int Keres(int[] A, int x) {
    int i=0;
    while (i<A.lenght && A[i]!=x) {
        i++;
    }
    return i;
}
\end{lstlisting}
\end{comment}

\begin{table}
    \centering
    \caption{Példa:}
    \begin{tabular}{cc}
        \includegraphics[height=2cm]{keres} & Value \\
    \end{tabular}
\end{table}
    \begin{comment}
\begin{wrapfigure}{l}{0.60\textwidth}
    \centering
        \includegraphics[height=4cm]{keres}
\end{wrapfigure}

Itt Keres$(A,n,x)$ futási ideje: $c_1+(i+1)c_2+ic_3+c_4$
\[T_{lj} = \min \{T(A,x):|x|=n\}\]
\[T_{lr} = \max\{T(A,x):|x|=n\}\]

\end{comment}

\end{document}
