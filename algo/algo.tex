%%%%%%%%%%%%%%%%%%%%%%%%%%%%% Define Article %%%%%%%%%%%%%%%%%%%%%%%%%%%%%%%%%%
\documentclass{article}
%%%%%%%%%%%%%%%%%%%%%%%%%%%%%%%%%%%%%%%%%%%%%%%%%%%%%%%%%%%%%%%%%%%%%%%%%%%%%%%

%%%%%%%%%%%%%%%%%%%%%%%%%%%%% Using Packages %%%%%%%%%%%%%%%%%%%%%%%%%%%%%%%%%%
\usepackage{geometry}
\usepackage{graphicx}
\usepackage{amssymb}
\usepackage{amsmath}
\usepackage{amsthm}
\usepackage{empheq}
\usepackage{mdframed}
\usepackage{booktabs}
\usepackage{lipsum}
\usepackage{graphicx}
\graphicspath{ {./images/} }
\usepackage{color}
\usepackage{psfrag}
\usepackage[fleqn]{nccmath}
\usepackage{comment}
\usepackage{pgfplots}
\usepackage{sidecap}
\usepackage{bm}
\usepackage{physics}
\usepackage{wrapfig}
\usepackage{listings}
\def\magyarOptions{suggestions=no} % nem tudom miért de kell ez a sor
\usepackage[magyar, english]{babel}
%%%%%%%%%%%%%%%%%%%%%%%%%%%%%%%%%%%%%%%%%%%%%%%%%%%%%%%%%%%%%%%%%%%%%%%%%%%%%%%

%%%%%%%%%%%%%%%%%%%%%%%%%% Page Setting %%%%%%%%%%%%%%%%%%%%%%%%%%%%%%%%%%%%%%%
\geometry{a4paper}


%%%%%%%%%%%%%%%%%%%%%%%%%% Define some useful colors %%%%%%%%%%%%%%%%%%%%%%%%%%
\definecolor{ocre}{RGB}{243,102,25}
\definecolor{mygray}{RGB}{243,243,244}
\definecolor{deepGreen}{RGB}{26,111,0}
\definecolor{shallowGreen}{RGB}{235,255,255}
\definecolor{deepBlue}{RGB}{61,124,222}
\definecolor{shallowBlue}{RGB}{235,249,255}
%%%%%%%%%%%%%%%%%%%%%%%%%%%%%%%%%%%%%%%%%%%%%%%%%%%%%%%%%%%%%%%%%%%%%%%%%%%%%%%

%%%%%%%%%%%%%%%%%%%%%%%%%% Define an orangebox command %%%%%%%%%%%%%%%%%%%%%%%%
\newcommand{\orangebox}[1]{\fcolorbox{ocre}{mygray}{\hspace{1em}#1\hspace{1em}}}
%%%%%%%%%%%%%%%%%%%%%%%%%%%%%%%%%%%%%%%%%%%%%%%%%%%%%%%%%%%%%%%%%%%%%%%%%%%%%%%

%%%%%%%%%%%%%%%%%%%%%%%%%%%% English Environments %%%%%%%%%%%%%%%%%%%%%%%%%%%%%
\newtheoremstyle{mytheoremstyle}{3pt}{3pt}{\normalfont}{0cm}{\rmfamily\bfseries}{}{1em}{{\color{black}\thmname{#1}~\thmnumber{#2}}\thmnote{\,--\,#3}}
\newtheoremstyle{myproblemstyle}{3pt}{3pt}{\normalfont}{0cm}{\rmfamily\bfseries}{}{1em}{{\color{black}\thmname{#1}~\thmnumber{#2}}\thmnote{\,--\,#3}}
\theoremstyle{mytheoremstyle}
\newmdtheoremenv[linewidth=1pt,backgroundcolor=shallowGreen,linecolor=deepGreen,leftmargin=0pt,innerleftmargin=20pt,innerrightmargin=20pt,]{theorem}{Theorem}[section]
\theoremstyle{mytheoremstyle}
\newmdtheoremenv[linewidth=1pt,backgroundcolor=shallowBlue,linecolor=deepBlue,leftmargin=0pt,innerleftmargin=20pt,innerrightmargin=20pt,]{definition}{Definition}[section]
\theoremstyle{myproblemstyle}
\newmdtheoremenv[linecolor=black,leftmargin=0pt,innerleftmargin=10pt,innerrightmargin=10pt,]{problem}{Problem}[section]
%%%%%%%%%%%%%%%%%%%%%%%%%%%%%%%%%%%%%%%%%%%%%%%%%%%%%%%%%%%%%%%%%%%%%%%%%%%%%%%

%%%%%%%%%%%%%%%%%%%%%%%%%%%%%%% Plotting Settings %%%%%%%%%%%%%%%%%%%%%%%%%%%%%
\usepgfplotslibrary{colorbrewer}
\pgfplotsset{width=8cm, compat=1.9}
%%%%%%%%%%%%%%%%%%%%%%%%%%%%%%%%%%%%%%%%%%%%%%%%%%%%%%%%%%%%%%%%%%%%%%%%%%%%%%%

%%%%%%%%%%%%%%%%%%%%%%%%%%%%%%% Title & Author %%%%%%%%%%%%%%%%%%%%%%%%%%%%%%%%
\title{Cím}
\author{Princzes
Barnabás}
%%%%%%%%%%%%%%%%%%%%%%%%%%%%%%%%%%%%%%%%%%%%%%%%%%%%%%%%%%%%%%%%%%%%%%%%%%%%%%%


\begin{document}
\selectlanguage{magyar}
\maketitle
\begin{abstract}
	Lent hagytam a füzetem és jövő héten doga szóval time to kick ass!
	Írjunk egy jó kis jegyzetet.
\end{abstract}
\section{Alapok}
Van a \{$B$\}$A$\{$K$\} azaz a Bemeneti
feltétel (egy logikai állítás) Algoritmus és Kimeneti feltétel
(szintén egy logikai állítás).
\\
Legyen $A$ algoritmus. $e_1,\ldots,e_m$ elemi műveletek az algoritmusban
és $t_i$ az edott $e_i$-hez tartozó időigény. Az algoritmus tényleges futási ideje
$T(A,x)$ ahol x egy bemenet és a bemenet mérete $|x|$ például
(tömb, halmaz\ldots) esetében az elemek száma.
Az $e_i$ és $t_i$ és $|x|$ együtt a bonyolultság mértéke.

\section{Esetek}
\begin{itemize}
	\item Legjobb eset: \[T_{lj} = \min \{T(A,x):|x|=n\}\]
	\item Legrosszabb eset: \[T_{lr} = \max\{T(A,x):|x|=n\}\]
	\item Átlagos eset: Legyen $\Pr(x)$ annak
	      a valószínűsége, hogy épp $x$ lesz
	      $A$ algoritmus bemenete, ekkor:
	      \\ \[T_a(A,n) = \sum_{|x|=n}\Pr(x)T(A,x)\]
\end{itemize}
\begin{center}
	\includegraphics[height=4cm]{keres}
	\\
	Itt Keres$(A,n,x)$ futási ideje: $c_1+(i+1)c_2+ic_3+c_4$
\end{center}


	$$T_{lj} = c_1+c_2+c_4 = O(1)$$
	$$T_{lr} = c_1+(n+1)c_2+nc_3+c_4=(c_2+c_3)n+c_1+c_2+c_4=O(n)$$
    Az átlagos eset számításához vezessünk be pár fogalmat:\\
    Legyen $B_0$ a legjobb bemeneti eset amikor is egyből megtaláljuk a keresett elemet\\
    Legyen $B_n$ amikor nem találjuk meg.\\
    Minden lehetséges bemenet $\in B_i\ |\ i=0,\ldots,n-1$ \\
    Legyen $D$ az Integer összes lehetséges értéke (nagy szám).\\ 
    Annak hogy az keresendő számot pont kiszúrjuk a valószínűsége: $p=\frac{1}{D}$\\
    Annak, hogy egy olyat találunk amelyik nem a keresett szám: $q=1-\frac{1}{D}$\\
    $Pr((A,n,x)\in B_i) = q^ip, i=0,\ldots,n-1$\\ 
    Itt $q^i$ jelöli, hogy hányszor nem találtuk meg és $p$ amikor megtaláltuk.
    Az utolsó lehetséges esetnél viszont csak $q$ eset fordul elő az pedig $n$-szer
    ugyanis végigmegyünk és nem találjuk meg a keresett elemet.
    $Pr((A,n,x)\in B_n) = q^n, $\\ 
	
    $$T_a(n) = \sum_{i=0}^n  Pr((A,n,x)\in B_i)(c1 + (i+1)c2 +ic3 +c4)$$
    Az átlagos esetet úgy kapjuk meg, ha minden lehetséges bemenet futási idejét 
    megszorozzuk az adott bemenet valószínűségével és az eseteket összeadjuk.\\
    A $B_n$ bemenetre vonatkozó eset specifikus így emeljük ki a többi közül:
    $$T_a(n) = q^n(c1 + (n+1)c2 +nc3 +c4)+\sum_{i=0}^{n-1} q^ip(c1 + (i+1)c2 +ic3 +c4)$$
    Mivel $q^n\to0$ (0-hoz tart) ezért felfele kerekítjük 1-re.
    Mivel a $\Sigma$ alatt mindent beszorzunk 
    $p(\ldots)$-al ahol a zárojelben $i<n$ így azt ki is emelhetjük ugyanúgy felfele 
    kerekítve $i$-t és $i+1$-et kicserélve $n$-re.
    $$T_a(n)\lessapprox\hat{T_a(n)}=(c1+(i+1)c2+ic3+c4)+(c_1+n(c_2+c_3)+c_4)p\sum_{i=0}^{n-1}q^i$$
    A $\Sigma$ tag egyenlő a mértani sorozat összegképletével, behelyettesítjük:
    $$\hat{T_a(n)}=(c1+(i+1)c2+ic3+c4)+(c_1+n(c_2+c_3)+c_4)p\frac{q^n-1}{q-1}$$
    Mivel $p=1-q$ ezért átalakítunk:
    $$\hat{T_a(n)}=(c1+(i+1)c2+ic3+c4)+(c_1+n(c_2+c_3)+c_4)(1-q^n)$$
    Mivel $(1-q^n)\to1$ megint felfele kerekítünk, majd összevonunk.
    $$\hat{T_a(n)}\lessapprox\hat{\hat{T_a(n)}}=(2c_2+2c_3)n+2c_1+c_2+2c_4$$
    Itt pedig $n$ a domináns tag szóval: 
    $$T_a(n)=O(n)$$
\end{document}
