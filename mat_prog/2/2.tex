\documentclass[12pt,a4paper]{article}
\usepackage{t1enc}
\usepackage[utf8]{inputenc}
\usepackage[mathscr]{eucal}
\usepackage[magyar]{babel}
\usepackage{amssymb}
\usepackage{amsthm}
\usepackage{graphics}
\usepackage{geometry}
\usepackage{amsxtra}
\usepackage{multirow}
\usepackage{multicol}
\usepackage{enumerate}
\usepackage{tabularx}
\usepackage{hhline}

\geometry{top=15mm,left=20mm,right=20mm,bottom=8mm}

\begin{document}
\pagestyle{empty}
\noindent

Matematikai programcsomagok
Gyakorló feladatsor
2. hét

\begin{enumerate}
    \item \textbf{Feladat.} Készítse el az alábbi táblázatot! \par
    \begin{center}
        \begin{tabular}{|c|c|c|}
            \hline
            $x\mapsto 3x$ & 
            $ x \in \left[-2;3\right] $ & 
            $x+3y\lneqq 4$ \\
            \hline
            $\left[4;4\right] \cap \left[-1;0\right] = \emptyset$ & 
            $ a \pm b $ & 
            $\alpha+\beta=\gamma$\\
            \hline
        \end{tabular}
    \end{center}

    \item \textbf{Feladat.} Adja meg a számsorozat határértékének definícióját (formálisan)! \par
    A $(x_n)$ valós számsorozat konvergens, ha $\exists$ olyan $x$ valós szám, hogy $\forall\varepsilon > 0$ (valós)
    számhoz található olyan $n_0 \in \mathbb{N}$ küszöbszám, hogy ha $n>n_0$,akkor 
    $\left|x_n-x\right|<\varepsilon$. Ekkor
    ezt az $x$ értéket a sorozat határértékének hívjuk.

    \item \textbf{Feladat.} Készítse el az alábbi táblázatot! 
    \begin{center}
        \begin{tabular}{|c|c|}
            \hline
            $\lim\limits_{x \to x_0} f(x)$ &$\lim_{x \to x_0} f(x)$\\
            \hline
            $\sum\limits_{n=0}^\infty a_n$&$\sum_{n=0}^{\infty}a_n$\\
            \hline
            $\displaystyle\int_{1}^{e^2}$&$\sum_{n=0}^{\infty}a_n$\\
            \hline
        \end{tabular}
    \end{center}

    \item \textbf{Feladat.} Készítse el az alábbi táblázatot!

    \item \textbf{Feladat.} Készítse el az alábbi feladatokat megoldásokkal együtt!

          \vfill\eject
    \item \textbf{Feladat.} Melyik konvergens az alábbi sorozatok közül, ha konvergens, akkor számolja a határértékét?

    \item \textbf{Feladat.} Írja fel az alábbi függvény $(2;3)$ pontjához húzott érintõjének egyenletét! \par
\end{enumerate}

\vspace{0.2cm}
\begin{center}
    \textit{Jó munkát!}
\end{center}

\end{document}