\documentclass[12pt,a4paper]{article}
\usepackage{t1enc}
\usepackage[latin2]{inputenc}
\usepackage[mathscr]{eucal}
\usepackage[magyar]{babel}
\usepackage{amssymb}
\usepackage{amsthm}
\usepackage{graphics}
\usepackage{geometry}
\usepackage{amsxtra}
\usepackage{multirow}
\usepackage{multicol}
\usepackage{enumerate}
\usepackage{tabularx}
\usepackage{hhline}

\geometry{top=15mm,left=20mm,right=20mm,bottom=8mm}

\begin{document}
\pagestyle{empty}
\noindent

Matematikai progsrasdamcsomagok
Gyakorló feladatsor
2. hét

\begin{enumerate}
    \item \textbf{Feladat.} Készítse el az alábbi táblázatot!

    \item \textbf{Feladat.} Adja meg a számsorozat határértékének definícióját (formálisan)!

    \item \textbf{Feladat.} Készítse el az alábbi táblázatot!

    \item \textbf{Feladat.} Készítse el az alábbi táblázatot!

    \item \textbf{Feladat.} Készítse el az alábbi feladatokat megoldásokkal együtt!

          \vfill\eject
    \item \textbf{Feladat.} Melyik konvergens az alábbi sorozatok közül, ha konvergens, akkor számolja a határértékét?

    \item \textbf{Feladat.} Írja fel az alábbi függvény $(2;3)$ pontjához húzott érintõjének egyenletét! \par
\end{enumerate}

\vspace{0.2cm}
\begin{center}
    \textit{Jó munkát!}
\end{center}

\end{document}