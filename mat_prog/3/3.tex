\documentclass[12pt,a4paper]{article}
\usepackage{t1enc}
\usepackage[utf8]{inputenc}
\usepackage[mathscr]{eucal}
\usepackage[magyar]{babel}
\usepackage{amssymb}
\usepackage{amsthm}
\usepackage{graphics}
\usepackage{geometry}
\usepackage{amsxtra}
\usepackage{multirow}
\usepackage{multicol}
\usepackage{enumerate}
\usepackage{tabularx}

\geometry{top=15mm,left=20mm,right=20mm,bottom=8mm}

\begin{document}
\pagestyle{empty}
\noindent

Matematikai programcsomagok
Gyakorló feladatsor
3.\ hét

\begin{enumerate}
    \item \textbf{Feladat.} Végezze el az $f(x)=x^3+2x^2-x-2$ függvény teljes vizsgálatát! \par

          \textbf{Megoldás}
          \begin{itemize}
              \item Értelmezési tartomány:
                    $dom(f)=\mathbb{R}$
              \item Zérushelyek:
                    \begin{gather*}
                        f(x)=x^3+2x^2-x-2=(x-1)(x+1)(x+2) \\
                        \Downarrow                        \\
                        x_0=1,x_1=-1,x_2=-2
                    \end{gather*}

              \item Határértékszámítás $-\infty$-ben és $\infty$-ben:

                    \begin{gather*}
                        \lim\limits_{x \to -\infty} x^3+2x^2-x-2=-\infty\\
                        \lim\limits_{x \to \infty} x^3+2x^2-x-2=\infty
                    \end{gather*}

              \item Menettulajdonságok az elsõ derivált segítségével:

                    \begin{gather*}
                        f'(x)=3x^2+4x-1=0\\
                        \Downarrow\\
                        x_1=-\frac{2}{3}+\frac{\sqrt{7}}{3}\approx-1,549,
                        \ x_2=-\frac{2}{3}-\frac{\sqrt{7}}{3}\approx-0,215
                    \end{gather*}
                    \[
                        \begin{array}{|c|c|c|c|c|c|}
                            \hline
                                  & x<x_1    & x=x_1 & x_1<x_<x_2 & x=x_2 & x<x_2    \\
                            \hline
                            f'(x) & +        & 0     & -          & 0     & +        \\
                            \hline
                            f(x)  & \nearrow & \max  & \searrow   & \min  & \nearrow \\
                            \hline
                        \end{array}
                    \]
              \item Görbületi viszonyok a második derivált segítségével:


                    \begin{gather*}
                        f''(x)=6x+4=0\\
                        \Downarrow\\
                        x_3=-\frac{2}{3}
                    \end{gather*}
                    \[
                        \begin{array}{|c|c|c|c|}
                            \hline
                                   & x<x_3 & x=x_3          & x_3<x \\
                            \hline
                            f''(x) & +     & 0              & -     \\
                            \hline
                            f(x)   & \cup  & \text{inf. p.} & \cap  \\
                            \hline
                        \end{array}
                    \]
          \end{itemize}

          \vfill\eject
    \item \textbf{Feladat.} Számolja ki az $y=x^2-2x+2$ és az $y=14-x^2$ egyenletû görbék által közrezárt korlátos síkidom területét!\\
          \textbf{Megoldás}
          \begin{itemize}
              \item Metszéspontok megkeresése:
                    \begin{gather*}
                        x^2-2x+2=14-x^2\\
                        2(x^2-x-6)=0\\
                        \Downarrow\\
                        x_1=-3,\ x_2=2\\
                    \end{gather*}
              \item A terület kiszámítása:
                    \[
                        \displaystyle\int_{-3}^{2}
                        -2(x^2-x-6)\ \mathrm{d}x=\left[-\frac{2}{3}x^3+x^2+12x\right]^2_{-3}=\frac{95}{3}
                    \]
          \end{itemize}

\end{enumerate}

\vspace{0.2cm}
\begin{center}
    \textit{Jó munkát!}
\end{center}

\end{document}