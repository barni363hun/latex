\documentclass[12pt,a4paper]{article}
\usepackage{t1enc}
\usepackage[mathscr]{eucal}
\usepackage[magyar]{babel}
\usepackage{amssymb}
\usepackage{amsthm}
\usepackage{graphics}
\usepackage{geometry}
\usepackage{amsxtra}
\usepackage{multirow}
\usepackage{multicol}
\usepackage{enumerate}
\usepackage{tabularx}
\usepackage{hhline}

\geometry{top=15mm,left=20mm,right=20mm,bottom=8mm}

\begin{document}
\pagestyle{empty}
\noindent

\begin{center}
    \textbf{Matematikai programcsomagok}
    \par
    \textbf{Gyakorló feladatsor}
    \par
    \textbf{1. hét}
    \par
\end{center}

Készítse el az alábbi feladatokat egy tex kiterjesztésû fájlba a lehetõ legpontosabban.
\par
\begin{enumerate}
\item
\textbf{Feladat.}
Készítse el az alábbi felsorolást, számozott listát kétféleképpen (táblázat segítségével, hasábokra osztva)!
\begin{multicols}{2}
    \begin{itemize}
        \item Álmos
        \item Elõd
        \item Ond
        \item Kond
        \item Tas
        \item Huba
        \item Töhötöm
    \end{itemize}
    \begin{enumerate}[a.)]
        \item Hapci
        \item Tudor
        \item Vidor
        \item Szende
        \item Szundi
        \item Morgó
        \item Kuka
    \end{enumerate}
\end{multicols}
\begin{tabularx}{\textwidth}{xx}
\begin{itemize}

    \item Álmos
    \item Elõd
    \item Ond
    \item Kond
    \item Tas
    \item Huba
    \item Töhötöm
\end{itemize}
Hapci
Tudor
Vidor
Szende
Szundi
Morgó
Kuka
\end{tabular}
\item
\textbf{Feladat.} Formázza hasonlóra a megadott szöveget!
\begin{center}
    \Large
    \textbf{Váczi Mihály}
    \par
    \large
    \textit{Nem elég! (részlet)}
    \par
    \normalsize Nem elég a jóra vágyni: \par

    a jót \textbf{akarni} kell! \par
    És nem elég akarni: \par
    de \textit{tenni},  tenni kell! \par
\end{center}
\item
\textbf{Feladat.} Készítse el az alábbi táblázatot a következõ módokon (tabular illetve tabularx környezet)!
\par

Ec pec kimehecc
Holnap után bejöhecc
\par
\item
\textbf{Feladat.} Készítse el kétféleképpen az alábbi szövegelrendezést (tabulátorokkal és táblázattal)!
IF korán kelsz
THEN aranyat lelsz
ELSE nagyot alszol
\par
\item

\textbf{Feladat.} Készítse el az alábbi táblázatot!
\par

a b
c d
e
\par

\end{enumerate}
Jó munkát!

\end{document}