%%%%%%%%%%%%%%%%%%%%%%%%%%%%% Define Article %%%%%%%%%%%%%%%%%%%%%%%%%%%%%%%%%%
\documentclass{article}
%%%%%%%%%%%%%%%%%%%%%%%%%%%%%%%%%%%%%%%%%%%%%%%%%%%%%%%%%%%%%%%%%%%%%%%%%%%%%%%

%%%%%%%%%%%%%%%%%%%%%%%%%%%%% Using Packages %%%%%%%%%%%%%%%%%%%%%%%%%%%%%%%%%%
\usepackage{geometry}
\usepackage{graphicx}
\usepackage{amssymb}
\usepackage{amsmath}
\usepackage{amsthm}
\usepackage{empheq}
\usepackage{mdframed}
\usepackage{booktabs}
\usepackage{lipsum}
\usepackage{graphicx}
\graphicspath{ {./images/} }
\usepackage{color}
\usepackage{psfrag}
\usepackage{comment}
\usepackage{pgfplots}
\usepackage{bm}
\usepackage{physics}
\usepackage{listings}
<<<<<<< Updated upstream

\usepackage[T1]{fontenc}
\usepackage[utf8]{inputenc}

\input glyphtounicode
\pdfgentounicode=1
% \def\magyarOptions{suggestions=no} % nem tudom miért de kell ez a sor
% \usepackage[magyar, english]{babel}
=======
% \def\magyarOptions{suggestions=no} % nem tudom miért de kell ez a sor
% \usepackage[magyar, english]{babel}
\usepackage{cmap}
>>>>>>> Stashed changes
%%%%%%%%%%%%%%%%%%%%%%%%%%%%%%%%%%%%%%%%%%%%%%%%%%%%%%%%%%%%%%%%%%%%%%%%%%%%%%%

%%%%%%%%%%%%%%%%%%%%%%%%%% Page Setting %%%%%%%%%%%%%%%%%%%%%%%%%%%%%%%%%%%%%%%
\geometry{a4paper}


%%%%%%%%%%%%%%%%%%%%%%%%%% Define some useful colors %%%%%%%%%%%%%%%%%%%%%%%%%%
\definecolor{ocre}{RGB}{243,102,25}
\definecolor{mygray}{RGB}{243,243,244}
\definecolor{deepGreen}{RGB}{26,111,0}
\definecolor{shallowGreen}{RGB}{235,255,255}
\definecolor{deepBlue}{RGB}{61,124,222}
\definecolor{shallowBlue}{RGB}{235,249,255}
%%%%%%%%%%%%%%%%%%%%%%%%%%%%%%%%%%%%%%%%%%%%%%%%%%%%%%%%%%%%%%%%%%%%%%%%%%%%%%%

%%%%%%%%%%%%%%%%%%%%%%%%%% Define an orangebox command %%%%%%%%%%%%%%%%%%%%%%%%
\newcommand{\orangebox}[1]{\fcolorbox{ocre}{mygray}{\hspace{1em}#1\hspace{1em}}}
%%%%%%%%%%%%%%%%%%%%%%%%%%%%%%%%%%%%%%%%%%%%%%%%%%%%%%%%%%%%%%%%%%%%%%%%%%%%%%%

%%%%%%%%%%%%%%%%%%%%%%%%%%%% English Environments %%%%%%%%%%%%%%%%%%%%%%%%%%%%%
\newtheoremstyle{mytheoremstyle}{3pt}{3pt}{\normalfont}{0cm}{\rmfamily\bfseries}{}{1em}{{\color{black}\thmname{#1}~\thmnumber{#2}}\thmnote{\,--\,#3}}
\newtheoremstyle{myproblemstyle}{3pt}{3pt}{\normalfont}{0cm}{\rmfamily\bfseries}{}{1em}{{\color{black}\thmname{#1}~\thmnumber{#2}}\thmnote{\,--\,#3}}
\theoremstyle{mytheoremstyle}
\newmdtheoremenv[linewidth=1pt,backgroundcolor=shallowGreen,linecolor=deepGreen,leftmargin=0pt,innerleftmargin=20pt,innerrightmargin=20pt,]{theorem}{Theorem}[section]
\theoremstyle{mytheoremstyle}
\newmdtheoremenv[linewidth=1pt,backgroundcolor=shallowBlue,linecolor=deepBlue,leftmargin=0pt,innerleftmargin=20pt,innerrightmargin=20pt,]{definition}{Definition}[section]
\theoremstyle{myproblemstyle}
\newmdtheoremenv[linecolor=black,leftmargin=0pt,innerleftmargin=10pt,innerrightmargin=10pt,]{problem}{Problem}[section]
%%%%%%%%%%%%%%%%%%%%%%%%%%%%%%%%%%%%%%%%%%%%%%%%%%%%%%%%%%%%%%%%%%%%%%%%%%%%%%%

%%%%%%%%%%%%%%%%%%%%%%%%%%%%%%% Plotting Settings %%%%%%%%%%%%%%%%%%%%%%%%%%%%%
\usepgfplotslibrary{colorbrewer}
\pgfplotsset{width=8cm, compat=1.9}
%%%%%%%%%%%%%%%%%%%%%%%%%%%%%%%%%%%%%%%%%%%%%%%%%%%%%%%%%%%%%%%%%%%%%%%%%%%%%%%

%%%%%%%%%%%%%%%%%%%%%%%%%%%%%%% Title & Author %%%%%%%%%%%%%%%%%%%%%%%%%%%%%%%%
\title{Analízis}
\author{Princzes
Barnabás}
%%%%%%%%%%%%%%%%%%%%%%%%%%%%%%%%%%%%%%%%%%%%%%%%%%%%%%%%%%%%%%%%%%%%%%%%%%%%%%%


\begin{document}
% \selectlanguage{magyar}
\maketitle
\begin{abstract}
<<<<<<< Updated upstream
    Analízis jegyzet Krasznai pdf-jeidből asd
\end{abstract}
\section{Összetett függvény}
\subsection{Leírás}
$fog$ -vel szoktuk jelölni az összetett függvényt más néven kompozíciót.\\
$(fog)(x) = f(g(x))$ ilyenkor szokás szerint a belső függvénnyel
kezdünk szóval $x$-et viszi valahova $g(x)$ ezután pedig az így képezett $y$-okat vesszük
=======
    Analízis jegyzet Krasznai pdf-jeiből asd
\end{abstract}
\section{Összetett függvény}
\subsection{Leírás}
$fog$ -vel szoktuk jelölni daz összetett függvényt más néven kompozíciót.\\
$(fog)(x) = f(g(x))$ ilyenkor szokás szerint a belső függvénnyel 
kezdünk szóval $x$-et viszi valahova $g(x)$ ezután pedig az így képezett $y$-okat vesszük 
>>>>>>> Stashed changes
$f(x)$ bemenetének, ebből következik, hogy $fog$ értelmezési tartománya ($D(fog)$)
azok az elemek ahol $g$ és $g$ képe is értelmezve van:
$$D(fog)=\{x\in D(g)\:|\:g(x)\in D(f)\}$$
\subsection{Példa}
$$f(x)=3x+1,\;x\in [0,7]$$
$$g(x)=x^2-9,\;x\in [0,5]$$
\subsubsection{Értelmezési tartomány}
Értelmezési tartomány megadása a képlet szerint:\\
$D(g)$: $0\leq x\leq 5$\\
$D(f)$: \underline{$0\leq x^2-9\leq 7$}
$ => 9\leq x^2\leq 16 => 3\leq \abs{x}\leq 4 => $
\underline{$-4\leq x\leq -3$ vagy $3\leq x\leq 4$}\\
Csak ott értelmezhető $f$ ahol $g$ is:\\
$D(g) \cap D(f) = D(fog) = [3,4]$\\
\subsubsection{Hozzárendelési utasítás}
Egyszerűen $f$ függvényébe beletesszük $x$ helyére $g(x)$-et:\\
$3(x^2-9)+1 = 3x^2-26$
\subsubsection{Megoldás}
$(fog)(x) = 3x^2-26,\;x\in [3,4]$
\subsection{Megjegyzések}
Ha $D(g) \cap D(f) = \emptyset$ akkor $fog$ nem létezik.
\section{Inverz függvény}
\subsection{Leírás}
$f$ függvény invertálható, ha akárhogyan választunk az értékkészletéből
két elemet $\bigl(x_1,x_2\in D(f)\bigr)$ akkor képeik nem azonosak
$\bigl(f(x_1)\neq f(x_2)\bigr)$\\
$f^{-1}$ inverze $f$-nek, ha $R(f)$-et képezi $D(f)$-be.\\
Invertálhatóság leolvasható a függvény grafikonjáról is:
Ha tudunk vízszintes vonalat húzni úgy, hogy kétszer metsze el a függvényt,
akkor az nem invertálható.\\
\subsection{Példa}
$$f(x)=1-x^2,\;x\in (-\infty ,0)$$
\subsubsection{Létezik $f^{-1}$?}
Ha $f$ minden kölünböző $x$-ből különböző $f(x)$-be visz akkor két ugyanazon
$x$-ből ugyanoda kell vinnie.\\
Jelenleg következik-e a két kép egyezéséből a két kiinulási pont egyezése?\\
Bizonyítsuk, hogy: $1-x_1^2=1-x_2^2=>x_1=x_2$ ahol $x_1,x_2\in (-\infty,0)$:\\
Ha elkezdünk számolni $1-x_1^2=1-x_2^2=>x_1^2=x_2^2=>|x_1|=|x_2|$, ami csak negatív számokon
van értelmezve így csak a negatív $x$-eket kell behelyettesíteni:
$-x_1=-x_2=>x_1=x_2$\\
\subsubsection{Értelmezési tartomány}
<<<<<<< Updated upstream
$f^{-1}$ értelmezési tartománya $f$ értékkészlete. Elinulunk $D(f)$-ből
=======
$f^{-1}$ értelmezési tartománya = $f$ értékkészlete. Elinulunk $D(f)$-ből 
>>>>>>> Stashed changes
$x\in (-\infty ,0)=x<0$
és addig alakítjuk amíg $f$ függvény formáját fel nem veszi:\\
$x<0=>x^2>0=>-x^2<0=>1-x^2<1$\\
$D(f^{-1})$

\end{document}